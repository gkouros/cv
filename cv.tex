% FortySecondsCV LaTeX template
% Copyright © 2019-2020 René Wirnata <rene.wirnata@pandascience.net>
% Licensed under the 3-Clause BSD License. See LICENSE file for details.
%
% Please visit https://github.com/PandaScience/FortySecondsCV for the most
% recent version! For bugs or feature requests, please open a new issue on
% github.
%
% Contributors
% ------------
% * ifokkema
% * Bertbk
% * Hespe
%
% Attributions
% ------------
% * fortysecondscv is based on the twentysecondcv class by Carmine Spagnuolo
%   (cspagnuolo@unisa.it), released under the MIT license and available under
%   https://github.com/spagnuolocarmine/TwentySecondsCurriculumVitae-LaTex
% * further attributions are indicated immediately before corresponding code


%-------------------------------------------------------------------------------
%                             ADDITIONAL PACKAGES
%-------------------------------------------------------------------------------
\documentclass[
	a4paper,
	% showframes,
	% vline=2.2em,
	maincolor=cvblue,
	% sidecolor=gray!20,
	sectioncolor=cvblue,
	% subsectioncolor=orange,
	% itemtextcolor=black!80,
	% sidebarwidth=0.4\paperwidth,
	% topbottommargin=0.03\paperheight,
	% leftrightmargin=20pt,
	% profilepicsize=4.5cm,
	profilepicborderwidth=3.0pt,
	% profilepicstyle=profilecircle,
	% profilepiczoom=1.0,
	% profilepicxshift=0mm,
	% profilepicyshift=0mm,
	% profilepicrounding=1.0cm,
]{fortysecondscv}

% improve word spacing and hyphenation
\usepackage{microtype}
\usepackage{ragged2e}

% itemize items in multiple columns
\usepackage{multicol}

% uncomment in case you don't want any hyphenation
% \usepackage[none]{hyphenat}

% take care of proper font encoding
\ifxetexorluatex
	\usepackage{fontspec}
	\defaultfontfeatures{Ligatures=TeX}
%	\newfontfamily\headingfont[Path = fonts/]{segoeuib.ttf} % local font
\else
	\usepackage[utf8]{inputenc}
	\usepackage[T1]{fontenc}
%	\usepackage[sfdefault]{noto} % use noto google font
\fi

% enable mathematical syntax for some symbols like \varnothing
\usepackage{amssymb}

% bubble diagram configuration
\usepackage{smartdiagram}
\smartdiagramset{
	% default font size is \large, so adjust to harmonize with sidebar layout
	bubble center node font = \footnotesize,
	bubble node font = \footnotesize,
	% default: 4cm/2.5cm; make minimum diameter relative to sidebar size
	bubble center node size = 0.4\sidebartextwidth,
	bubble node size = 0.25\sidebartextwidth,
	distance center/other bubbles = 1.5em,
	% set center bubble color
	bubble center node color = maincolor!70,
	% define the list of colors usable in the diagram
	set color list = {maincolor!10, maincolor!40,
	maincolor!20, maincolor!60, maincolor!35, maincolor!45},
	% sets the opacity at which the bubbles are shown
	bubble fill opacity = 0.8,
}


%-------------------------------------------------------------------------------
%                            PERSONAL INFORMATION
%-------------------------------------------------------------------------------
%% mandatory information
% your name
\cvname{Georgios Kouros}
% job title/career
\cvjobtitle{AI Master Student}

%% optional information
% profile picture
\cvprofilepic{pics/cv_pic.jpg}

% NOTE: ordering in sidebar will mimic the following order
% date of birth
%\cvbirthday{May 9th, 1992}
% short address/location, use \newline if more than 1 line is required
\cvaddress{\href{https://www.google.com/maps/place/Leuven,+Belgium/@50.8841973,4.6353906,12z/data=!3m1!4b1!4m5!3m4!1s0x47c160d05ebbdf85:0x40099ab2f4d5690!8m2!3d50.8798438!4d4.7005176}{Leuven, Belgium}}
% phone number
\cvphone{+32 498 44 94 22}
% personal website
% email address
\cvmail{george.kouros.ece@gmail.com}
\cvcustomdata{\faLinkedin}{\href{http://linkedin.com/in/gkouros}{linkedin.com/in/gkouros}}
\cvcustomdata{\href{http://github.com/gkouros}{\faGithub}}{\href{http://github.com/gkouros}{github.com/gkouros}}
\cvsite{https://gkouros.github.io/}


% pgp key
%\cvkey{4096R/FF00FF00}{0xAABBCCDDFF00FF00}
% any other custom entry
\cvcustomdata{\faFlag}{Greek}

%-------------------------------------------------------------------------------
%                              SIDEBAR 1st PAGE
%-------------------------------------------------------------------------------
% add more profile sections to sidebar on first page
\addtofrontsidebar{

	\profilesection{About Me}
	\aboutme{I am an autonomous robots specialist, experienced in search-and-rescue, utility mapping and industrial robotics. I am always busy pursuing new ways to utilize artificial intelligence in order to create trully intelligent robots. Robots that add real value to society.}
	
	\graphicspath{{pics/flags/}}
	\profilesection{Languages}
		\pointskill{\flag{GR.png}}{Greek}{5}
		\pointskill{\flag{GB.png}}{English}{5}
		\pointskill{\flag{DE.png}}{German}{1}
	
	\profilesection{Expertise}
	\begin{figure}\centering
		\smartdiagram[bubble diagram]{
			\textcolor{white}{\textbf{Autonomous}}\\
			\textcolor{white}{\textbf{Robotics}},
			\textcolor{black!90}{ROS},
			\textcolor{black!90}{C++},
			\textcolor{black!90}{Python},
			\textcolor{black!90}{Navigation},
			\textcolor{black!90}{SLAM},
			\textcolor{black!90}{Computer}\\
			\textcolor{black!90}{Vision}
		}
	\end{figure}	
				

%	\profilesection{Technical Skills}
%		\begin{sidebarminipage}
%			\chartlabel{C/C++}
%			\chartlabel{Python}
%			\chartlabel{ROS}
%			\chartlabel{Gazebo}
%			\chartlabel{PCL}
%			\chartlabel{OpenCV}
%			\chartlabel{PyTorch}
%			\chartlabel{Keras}
%			\chartlabel{Tensorflow}
%			\chartlabel{CI/CD}
%			\chartlabel{Docker}
%			\chartlabel{Git}
%			\chartlabel{Linux}
%			\chartlabel{Blender}
%		\end{sidebarminipage}

%	\profilesection{Misc. Skills}
%		\begin{sidebarminipage}
%			\chartlabel{Vim Editor}
%			\chartlabel{\LaTeX}
%			\chartlabel{Microsoft Office}
%			\chartlabel{Arduino}
%			\chartlabel{RPi}			
%			\chartlabel{Electronics}
%			\chartlabel{PCB Design}
%			\chartlabel{Image / Video Editing}
%			\chartlabel{3D Modelling}
%			\chartlabel{Driver's License}
%		\end{sidebarminipage}

		
		

%	\profilesection{Hard Skills}
%		\skill{\faBalanceScale}{Sleeping almost all day}
%		\skill{\faSitemap}{Eating a lot of bamboo sprouts}
%		\skill{\faGraduationCap}{Relaxing rest of the day}

%	\profilesection{Soft Skills}
%		\pointskill{\faHome}{Looking Cute}{4}[4]
%			\skill[1.8em]{\faCompress}{No need to specif further}
%		\pointskill{\faChild}{Chillin' hard}{3}[4]
%			\skill[1.8em]{\faCompress}{On a tree}
%			\skill[1.8em]{\faCompress}{In the grass}


			
}


%-------------------------------------------------------------------------------
%                              SIDEBAR 2nd PAGE
%-------------------------------------------------------------------------------
%\addtobacksidebar{
%
%	\profilesection{Diagrams}
%	\begin{sidebarminipage}
%		\chartlabel{Bubble}
%		\chartlabel{Diagrams}
%		\chartlabel{with}
%		\chartlabel{proper}
%		\chartlabel{overflow}
%		\chartlabel{protection}
%		\chartlabel{for}
%		\chartlabel{labels}
%	\end{sidebarminipage}
%
%	\begin{figure}\centering
%		\smartdiagram[bubble diagram]{
%			\textcolor{white}{\textbf{Being a}} \\
%			\textcolor{white}{\textbf{Panda}}, % center bubble
%			\textcolor{black!90}{Eating},
%			\textcolor{black!90}{Sleeping},
%			\textcolor{black!90}{Rolling},
%			\textcolor{black!90}{Playing},
%			\textcolor{black!90}{Chilling}
%		}
%	\end{figure}
%
%	\chartlabel{Wheel Chart}
%
%	\wheelchart{3.7em}{2em}{%
%	20/3em/maincolor!50/Chill,
%	15/3em/maincolor!15/Play,
%	30/4em/maincolor!40/Sleep,
%	20/3em/maincolor!20/Eat
%	}
%
%	\profilesection{Barskills}
%	\barskill{\faSkyatlas}{Wearing asian rice hats}{60}
%	\barskill{\faImage}{Playing Chess}{30}
%	\barskill{\faMusic}{Playing the bamboo flute}{50}
%
%	\profilesection{Memberships}
%	\begin{memberships}
%		\membership[4em]{pics/logo.png}{PandaScience.net}
%		\membership[4em]{pics/logo.png}{Some longer text spanning over more than
%			only one line}
%	\end{memberships}
%}


%-------------------------------------------------------------------------------
%                         TABLE ENTRIES RIGHT COLUMN
%-------------------------------------------------------------------------------
\begin{document}

\makefrontsidebar

\cvsection{Education}
\begin{cvtable}[1.5]
	\vspace{0.3cm}
	\cvitem{2020 -- Now}{Advanced Master of Artificial Intelligence}{KU Leuven}
		{Track: Engineering and Computer Science (ECS)}
	\cvitem{2010 -- 2016}{Electrical and Computer Engineering Diploma (BSc \& MSc)}{Aristotle University of Thessaloniki}
		{Specialization: Electronics and Computers, GPA: 7.84/10}
	\cvitem{}{Diploma Thesis (10/10)}{}
		{Development of an Autonomous Robotic Ground Vehicle with a 4WS4WD
Kinematic Model and Implementation of a System for Autonomous Exploration in
Unknown Environments}
\end{cvtable}

\cvsection{Experience}
\vspace{-0.4cm}
\begin{cvtable}[3]
\vspace{-0.25cm}
\cvitem{2019 -- 2020}{Industry 4.0 Consultant}{Kapernikov CVBA}{Worked on computer vision based forklift localization, quality testing of industrial products, and industrial operation monitoring. Also, organized a company wide knowledge-sharing workshop on ROS, Visual Odometry and Visual SLAM.}
\vspace{-0.25cm}
\cvitem{2017 -- 2018}{Research Assistant}{ ITI -- CERTH}{Was responsible for the development of an autonomous robotic system for subsurface utility mapping as part of Project BADGER (EU Horizon 2020). Focused on the development of algorithms for autonomous navigation of a tractor-trailer UGV and autonomous subsurface mapping with a ground penetrating radar.}
\cvitem{2014 -- 2016}{Volunteer Robotics Engineer}{Pandora Robotics Team}{
As a member of a university robotics team, I had the opportunity to work on all the stages of development of an advanced autonomous UGV for search-and-rescue operations (hardware, sensor/actuator drivers, testing, simulation, control, localization, mapping, and navigation).}
\end{cvtable}

\cvsection{Publications}
\begin{cvtable}
	\vspace{0.25cm}
	\cvpubitem
	{3D Underground Mapping with a Mobile Robot and a GPR Antenna}
	{G. Kouros, E. Skartados, I. Kostavelis, D. Giakoumis and D. Tzovaras}
	{IEEE/RSJ International Conference on Intelligent Robots and Systems (IROS), Madrid, Spain}{2018}
	\vspace{0.2cm}

	\cvpubitem
	{Surface/subsurface mapping with an integrated rover-GPR system: A simulation approach}
	{G. Kouros, C. Psarras, I. Kostavelis, D. Giakoumis and D. Tzovaras}
	{2018 IEEE International Conference on Simulation, Modeling, and Programming for Autonomous Robots SIMPAR, Brisbane, Australia.}{2018}
	\cvpubitem
	{PANDORA Monstertruck: A 4WS4WD car-like robot for autonomous exploration in unknown environments}
	{G. Kouros, L. Petrou}
	{12th IEEE Conference on Industrial Electronics and Applications (ICIEA), Siem Reap, Cambodia}{2017}
\end{cvtable}


\cvsection{Awards}
\begin{cvtable}
	\cvitem{2015}{2nd Best in Class Autonomy}{Pandora Robotics Team}{Robocup Rescue 2015 competition in Hefei, China}
	\cvitem{2015}{Excellence Award}{Pandora Robotics Team}{Awarded by the Aristotle University of Thessaloniki for the Distinction in Robocup Rescue 2015 competition}
\end{cvtable}


\cvsection{Technical Skills}
%		\chartlabel{Gazebo}
%		\chartlabel{OpenCV}
%		\chartlabel{PCL}
%		\chartlabel{Tensorflow}
%		\chartlabel{Keras}
%		\chartlabel{CVAT}
%		\chartlabel{CI/CD}
%		\chartlabel{Docker}
%		\chartlabel{Git}
%		\chartlabel{Ubuntu}
%		\chartlabel{Arch-Linux}
%		\chartlabel{Blender}
%		\chartlabel{\LaTeX}
%		\chartlabel{Microsfot Office}
%		\chartlabel{Matlab}
\vspace{-0.3cm}
\begin{multicols}{5}
\begin{itemize}
  \item Linux
  \item OpenCV
  \item ROS
  \item PCL
  \item Gazebo
  \item Blender
  \item Git
  \item PyTorch
  \item smach
  \item \LaTeX
  \item Docker
  \item OMPL
  \item CI/CD
  \item CVAT
  \item Jira
\end{itemize}
\end{multicols}


%\cvsection{Extra-Curricular Activities}
%\begin{cvtable}
%	\cvitemshort{Relaxing}{Master the fine art of relaxing everywhere}
%	\cvitemshort{Music}{Playing the bamboo flute in the 1st Panda Orchestra}
%	\cvitemshort{Education}{Teaching young pandas to be more panda-like}
%\end{cvtable}

%\newpage
%\makebacksidebar
% \newgeometry{
% 	top=\topbottommargin,
% 	bottom=\topbottommargin,
% 	right=\leftrightmargin,
% 	left=\leftrightmargin
% }

%\cvsection{section}
%\cvsubsection{Subsection}
%\begin{cvtable}
%	\cvitem{<dates>}{<cv-item title>}{<location>}{<optional: description>}
%\end{cvtable}
%
%\cvsection{cvitem}
%\cvsubsection{Multi-line with longer description}
%\begin{cvtable}
%	\cvitem{date}{Description}{location}{Some longer and more detailed
%		description, that takes two lines of space instead of only one.}
%	\cvitem{date}{Description}{location}{Some longer and more detailed
%		description, that takes two lines of space instead of only one.}
%	\cvitem{date}{Description}{location}{Some longer and more detailed
%		description, that takes two lines of space instead of only one.}
%\end{cvtable}
%
%\cvsubsection{One-line without description}
%\begin{cvtable}
%	\cvitem{Award}{One-line description}{Sponsor}{}
%	\cvitem{Award}{One-line description}{Sponsor}{}
%	\cvitem{Award}{One-line description}{Sponsor}{}
%\end{cvtable}
%
%\cvsection{cvitemshort}
%\cvsubsection{One-line}
%\begin{cvtable}
%	\cvitemshort{Key}{Some further description}
%	\cvitemshort{Key}{Some further description}
%	\cvitemshort{Key}{Some further description}
%\end{cvtable}
%
%\cvsubsection{Multi-line with longer description}
%\begin{cvtable}
%	\cvitemshort{Key}{Some further description. Can fill even more than
%		only one single line while still keeping the correct indendation level.}
%	\cvitemshort{Key}{Some further description. Can fill even more than
%		only one single line while still keeping the correct indendation level.}
%	\cvitemshort{Key}{Some further description. Can fill even more than
%		only one single line while still keeping the correct indendation level.}
%\end{cvtable}
%
%\cvsection{cvpubitem}
%\begin{cvtable}
%	\cvpubitem{Publication title}{Authors}{Journal}{Year}
%	\cvpubitem{Publication title}{Authors}{Journal}{Year}
%	\cvpubitem{Publication title that is spanning over multiple lines and still
%		does not look too bad}{Authors}{Journal}{Year}
%\end{cvtable}

%\cvsignature

\end{document}
